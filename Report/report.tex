\documentclass[a4paper,10pt]{article}
\usepackage{hyperref}
\usepackage{pstricks}
\usepackage{tikz}
\usepackage{helvet}
\usepackage{ulem}
\usepackage{amsfonts}
\bibliographystyle{plain}
\pagestyle{empty}

% PDF settings
\hypersetup{%
  pdftitle={An Overview of Gentry's Fully Homomorphic Encryption Scheme}, %
  pdfauthor={Adam Risi and Ross Snider}, %
}

\title{An Overview of Gentry's Fully Homomorphic Encryption Scheme}
\author{Adam Risi \and Ross Snider}

\begin{document}
\maketitle
\titlepage
\tableofcontents
\pagebreak

\begin{abstract}
In a 1978 paper\cite{rad-priv} inspired by the multiplicative
homomorphic properties of the RSA cryptosystem, Rivest, Shamir and
Dertouzos detailed essential properties of what they called a “privacy
homomorphism” – a cryptographic scheme that would allow third party
processing of encrypted data without access to any plaintext or
private key information. While partial strides toward such a system
have been made over the past thirty years, most cryptographers
believed that no such scheme could actually exist. Craig Gentry of
Stanford University fully realized such a scheme in his 2009 PhD
dissertation “A Fully Homomorphic Encryption Scheme Over Ideal
Lattices.”\cite{gentry-lattice} In this paper, we describe a similar
(fully homomorphic) scheme over the integers (also invented by
Gentry), which remains conceptually identical to the original lattice
scheme but with the added benefit that it is much simplier.
\end{abstract}

\section{Introduction}
In his 2009 paper ``Fully Homomorphic Encryption over the
Integers''\cite{gentry-integers}, Craig Gentry made two important
contributions to the field. First, he defined a partially homomorphic
scheme which was interesting in its own right. This scheme was robust
enough to allow third party contributions of a nearly arbitrary number
of additions and many multiplications, beating out the
Boneh-Goh-Nissim cryptosystem, which remained a leader in the field
for quite some time.

Unfortunately, Gentry's partially homomorphic scheme suffered from the
fact that as computations are performed on the ciphertexts, noise is
introduced in the output ciphertexts. This limited the malleability of
the scheme – after some number of computations, the error vectors
would grow too large and decryption would become invalid.

Gentry realized that if he could find some way to eliminate the
growing noise implicit in the scheme, he could modify his original
scheme to allow an arbitrary number of additions and
multiplications. In turn, this would allow him to perform any
computable function on encrypted data.

Aware that successfully decrypting and subsequently re-encrypting a
given ciphertext would eliminate its noise vector, Gentry asked 'why
not allow the third party to decrypt the ciphertexts homomorphically?'
The third party would be able to remove the growing noise from
ciphertexts. At the same time, asking the third party to do the
decryption homomorphically would maintain the privacy of the private
key.

If the noise introduced by a homomorphic evaluation of the decryption
function is less than what the scheme is capable of handling, then the
scheme can chain together homomorphic evaluations of decryption with
homomorphic evaluations of the target computable functions, resulting
in a scheme that can handle a computable function of any complexity.

\section{History}
While a Fully Homomorphic Encryption scheme is new, the idea is
not. The inspiration for a fully homomorphic encryption system came
from an interesting property in RSA.

\begin{eqnarray}
c = m ^{e} \bmod n\\
m = c ^{d} \bmod n 
\end{eqnarray}

If we multiply the ciphertext $c$ by $y ^{e}$, we can see that the
decryption of $ cy^{e} $ will yield $my$, or the plaintext multiplied
by the value y. This \textit{partial homomorphism} lead to greater
investivation by Rivest, Adleman, and Dertouzos.

Proposed in 1978 by Rivest, Adleman, and Dertuzos, ``privacy
homomorphisms'' are the first known example of what came to be called
``fully homomorphic encryption.'' Rivest, Adleman, and Dertuzos
proposed a thought experiment: Imagine a loan company who stored its
data at an off site location. Since the data about loans is sensitive,
the loan company decided to have all of its data encrypted. Now, a
problem becomes apparent: if the loan data is stored off site, then
how would the loan company be able to query it? Rivest, Adleman, and
Dertuzos proposed that ``privacy homomorphisms'' could be used, a
theoretical encryption system where data could be stored, and queried,
in a completely encrypted environment.

Between 1978 and today, a number of other schemes have been shown to
be partially homomorphic. Paillier, Boneh-Goh-Nissim, and ElGamal all
exhibited partial homomorphism over some operations.

\section{Background}

\subsection{Multiplicative Homomorphism in RSA}
In RSA, the encryption function $\varepsilon$ and decryption function $\delta$ provide a multiplicative homomorphism. First, recall:
\begin{eqnarray}
\varepsilon_{RSA}(m) &=& m^e \bmod n\\
\delta_{RSA}(c) &=& c^d \bmod n\\
de &=& 1 \bmod \phi(n)\\
m^{de} \bmod n &=& m \bmod n\\
\\
\varepsilon_{RSA}(x) \varepsilon_{RSA}(y) = 
\end{eqnarray}


\subsection{Partially Homomorphic and Fully Homomorphic Encryption}

\subsection{Constructing a Boolean Circuit}


\section{Gentry's Homomorphic Encryption Scheme}

\subsection{Partially Homomorphic Encryption}
To begin, Gentry proposes a simple partially homomorphic encryption
scheme over the integers:

Key Generation: The key, $p$, is a randomly generated odd integer.

Encryption: To encrypt a single bit $ m \in \{0,1\} $, pick a
ciphertext integer whose residue mod $ p $ has the same parity as
$m$. This ciphertext integer can be described as $ c = pq + m $, where
$ p $ is the private key, and $q$ is a $\lambda$ bit integer.

Decryption: Decryption under this scheme is simple, the given a
ciphertext $c$, $ m = ( c \bmod p) \bmod 2 $.

We can show that this is homomorphic by looking showing the
multiplication and addition of two cithpertexts, $ \varepsilon(x) $
and $ \varepsilon(y) $.

\paragraph{Multiplication}
\begin{eqnarray*}
\end{eqnarray*}

\paragraph{Addition}

\subsection{Fully Homomorphic Encrpytion}

\subsection{Construction}
Gentry proposes a fully homomorphic encryption scheme that relies on
two different ``difficult problems'' for security, the
\textbf{Approximate Greatest Common Divisor} problem, and the
\textbf{Sparse Subset Sum} problem.

\subsubsection{Approximate Greatest Common Divisor}
The approximate GCD problem is, given a set of $ n _{i} + pq_{i} $,
determine p. The best known algorithm is exponential time in the size
of inputs.


\subsubsection{Sparse Subset Sum}
The sparse subset sum problem is a variant of the common subset sum
problem. In the classic Subset Sum problem, we are given a set $
\mathbb{s} $ and a target value $t$, and are asked for a subset of $
\mathbb{s} $ whose \textit{sum} is equal to $t$. In the \textit{Sparse
  Subset Sum} problem, the simple addition that the cardinality of the
answer set must be much smaller than the cardinality of $ \mathbb{s}
$. This problem, and its parent problem (Subset Sum), have been shown
to be intractable reasonably sized $ \mathbb{s} $.

\subsection{Bootstrapping}

\section{Fully Homomorphic Encryption in the Future}

\pagebreak
\bibliography{report}
\end{document}


